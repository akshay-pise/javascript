1. what is function..?
 function is block of code or function is used wrap the code.
 syntax: 
      function functionName(){
                                      statement/condition
                                }
                                
2. type of function.?
-> function has two type of function 
            1. simple function      
            -> a function has no parameter
                  syntax:- function functionname(){
                        condition/statement;
                  }

                  you can write with two Way
                        I) simple function with no return value.
                        eg.
                        var num1=8, num2=5, addition;
                        function funname(){
                              addition=num1+num2;
                        }
                        console.log(funname);

                        II) simple function with return value.
                        eg. 
                        var num1=8, num2=5,;
                        function funname(){
                              return num1+num2;
                        }
                        console.log(funname);

            2. function with parameter
            -> a function with parameter has pass the value with  in function.
            note:- if you have pass nth number of  parameter in function then you pass the n number of value of parameter
                        ( function functionNm (parameter1,.... ,parameterN){

                        }
                        

                        )
            syntax;
            function functionNm(parameter1,parameter2)
            {
                  parameter1
                  parameter2
            } 
            functionNm(value1, value2);

            you can write with two way
            I) parameter function with no return value.
            eg. function funame(add1, add2){
                  var result=add1+add2;   
                  console.log(result);
            }
            //call
            funame(20,30);

            II) parameter function with return value.
            eg.
            function funame(add1, add2){
                  return add1+add2;
            }
            
            //call
            funame(20,30);
            console.log(funame);
3.how to create html element using js dyanamicly.?
-there are step to create 
      first we need to create element so you can acess document.getElementById("parent/target-element-name") and store in one variable.
      thensecond we need element name which one create like document.createElement("create-element-name)
      and last appendChild this for create element or merge two step using appendChild keyword (ul.appendChild(li);)
      (target element -> createElement-name ->appendChild)
      
      -i.e. var ul = document.getElementById("myUL");
                  "myUl" -this location where you want to create element. and i store in ul variable
            var li = document.createElement("li");
                  "li" -this is element name which creating and sote in li variable.
            ul.appendChild(li); 
                  appendChild- this is create child element(li) inside to  parent element(ul)


